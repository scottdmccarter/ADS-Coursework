\documentclass[12pt]{article}%
\usepackage{amsfonts}
\usepackage{fancyhdr}
\usepackage{comment}
\usepackage[a4paper, top=2.5cm, bottom=2.5cm, left=2.2cm, right=2.2cm]%
{geometry}
\usepackage{times}
\usepackage{amsmath}
\usepackage{changepage}
\usepackage{amssymb}
\usepackage{graphicx}%
\setcounter{MaxMatrixCols}{30}
\newtheorem{theorem}{Theorem}
\newtheorem{acknowledgement}[theorem]{Acknowledgement}
\newtheorem{algorithm}[theorem]{Algorithm}
\newtheorem{axiom}{Axiom}
\newtheorem{case}[theorem]{Case}
\newtheorem{claim}[theorem]{Claim}
\newtheorem{conclusion}[theorem]{Conclusion}
\newtheorem{condition}[theorem]{Condition}
\newtheorem{conjecture}[theorem]{Conjecture}
\newtheorem{corollary}[theorem]{Corollary}
\newtheorem{criterion}[theorem]{Criterion}
\newtheorem{definition}[theorem]{Definition}
\newtheorem{example}[theorem]{Example}
\newtheorem{exercise}[theorem]{Exercise}
\newtheorem{lemma}[theorem]{Lemma}
\newtheorem{notation}[theorem]{Notation}
\newtheorem{problem}[theorem]{Problem}
\newtheorem{proposition}[theorem]{Proposition}
\newtheorem{remark}[theorem]{Remark}
\newtheorem{solution}[theorem]{Solution}
\newtheorem{summary}[theorem]{Summary}
\newenvironment{proof}[1][Proof]{\textbf{#1.} }{\ \rule{0.5em}{0.5em}}

\newcommand{\Q}{\mathbb{Q}}
\newcommand{\R}{\mathbb{R}}
\newcommand{\C}{\mathbb{C}}
\newcommand{\Z}{\mathbb{Z}}

\title{Algorithms & Data Structures 2018/19 Coursework}
\author{rwcj49}
\date{  }
\begin{document}

\maketitle


\section*{Q4}
\subsection*{(a)}
$2x^4$ is $\mathcal{O}(x^3 + 3x +2) \xrightarrow{}$ \textit{False} \\
\bigskip
\begin{quote}
PROOF:\\

\skip
Simplify $\mathcal{O}(x^3 + 3x +2)$ to  $\mathcal{O}(x^3)$, ignoring low order terms\\

For $x > 0$,\hspace{10px} $x^4 \geq  x^3$
\begin{quote}
There exists no $k>0$ \& $c >0 $ such that $x^4 \leq c \cdot x^3$, when $x \geq k$ \\
Therefore the statement is false.

\end{quote}
\end{quote}




\subsection*{(b)}
$4x^3 +2x^2 \cdot logx + 1$ is $\mathcal{O}(x^3) \xrightarrow{}$ \textit{True} \\
\bigskip
\begin{quote}
PROOF:
\begin{quote}
Can rewrite as:
    $$f_1(x) + f_2(x) + f_3(x)$$
    Where:\\
    $f_1(x) = 4x^3$ is $\mathcal{O}(x^3)$, taking $c = 4$ and any $k > 0$\\
    $f_2(x) = 2x^2 \cdot \log x$ is $\mathcal{O}(x^2\cdot \log x)$, taking $c = 2$ and any $k > 0$\\
    $f_3(x) = 1$ is $\mathcal{O}(1)$, taking $c = 1$ and any $k$\\
    Therefore, by sum rule, $4x^3 +2x^2 \cdot logx + 1$ is $\mathcal{O}(x^3)$ and so the statement is true\\
\end{quote}
\end{quote}
\subsection*{(c)}
$3x^2 +7x + 1$ is $\omega(x\cdot logx) \xrightarrow{}$ \textit{True} \\
\bigskip
\begin{quote}
PROOF:\\
$$3x^2 +7x + 1 = \omega(x\cdot logx) \implies x\cdot logx = o(3x^2 +7x + 1)$$
\begin{center}
    $\lim_{x \to \infty} \dfrac{x\cdot \log x}{3x^2 + 7x +1} \hspace{5px}  =\hspace{5px}\lim_{x \to \infty} \dfrac{\dfrac{\log x}{x}}{3 + \dfrac{7}{x} + \dfrac{1}{x^2}}\hspace{5px} = \dfrac{0}{3} \hspace{5px}=\hspace{5px} 0$
\end{center}
Therefore, $x\cdot logx = o(3x^2 +7x + 1)$ is true, and so too is the original statement of $3x^2 +7x + 1 = \omega(x\cdot logx) $
\end{quote}
\subsection*{(d)}
$x^2 +4x$ is $\Omega(x \cdot logx) \xrightarrow{}$ \textit{True} \\
\bigskip
\begin{quote}
PROOF:\\


\skip
For $k=2,c=1$:\\
\begin{quote}
$x^2 + 4x \geq c \cdot x \cdot \log x$\\
$= 2^2 + 4(2) \geq 1 \cdot 2 \cdot 1$\\
$12 \geq 2$\\
This is true and therefore so too is the original statement.
\end{quote}
\end{quote}
\subsection*{(e)}
$f(x) + g(x)$ is $\Theta (f(x)\cdot g(x)) \xrightarrow{}$ \textit{True} \\
\bigskip
\begin{quote}
PROOF:\\
    $deg(f(x)+g(x)) \leq deg(f(x)g(x))$\\


\end{quote}


\section*{Q5}
\subsection*{(a)}
$T(n) = 9T(n/3) + n^2$\\
\bigskip
$\implies T(n) = \Theta(n^2 \log n)$ (Case 2)

\subsection*{(b)}
$T(n) = 4T(n/2) + 100n$ \\
\bigskip
$\implies T(n) = \Theta(n^2)$ (Case 1)


\subsection*{(c)}
$T(n) = 2^nT(n/2) + n^3$\\
\textbf{Cannot use Master Theorem:} \\
\noindent
To use Master Theorem, recurrence must be of the form:
$$T(n) = aT(n/a) + f(n)$$ Where $a \geq 1$ and $b \geq 1$\\
Here $a$ is $2^n$, not a constant .
\subsection*{(d)}
$T(n) = 3T(n/3) + c\cdot n$\\
\bigskip
$\implies T(n) = \Theta(n^2 \log n)$ (Case 2)

\subsection*{(e)}
$T(n) = 0.99T(n/7) + 1/(n^2)$ \\
\textbf{Cannot use Master Theorem:} \\
\noindent
To use Master Theorem, recurrence must be of the form:
$$T(n) = aT(n/a) + f(n)$$ Where $a \geq 1$ and $b \geq 1$\\
$a < 1$ therefore not compatible with Master Theorem.
\section*{Q6}
\subsection*{(a)}
See rwcj49\_q6a.py
\subsection*{(b)}
MergeSort has worst complexity of $\mathcal{O} (n \cdot \log n)$\\
SelectionSort has worst complexity $\mathcal{O}(n^2)$ \\

\noindent
While Selectionsort has worse complexity than MergeSort for large arrays, in this algorithm, SelectionSort only ever acts on small arrays (length $\leq$ 4), and so the overall worst case input will be that which gives highest complexity for MergeSort.\\
This worst case MergeSort is that which involves most comparisons\\
e.g.
\begin{quote}
    $A = [8,16,4,12,6,14,2,10,7,15,3,11,5,13,1,9]$
\end{quote}
    This array will require swaps at every possible stage, giving the highest possible complexity for an array of this length.
\subsection*{(c)}
See rwcj49\_q6c.py


\end{document}
